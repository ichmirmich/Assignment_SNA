\documentclass[12pt, titlepage=true, toc=bib]{scrartcl}


%Deutsche Silbentrennung
%Trennung von Wörtern mit Umlauten
%Deutsche Sprachregeln
\usepackage[utf8]{inputenc}
\usepackage[T1]{fontenc}
\usepackage[english]{babel}

%Schriftart
\usepackage[bitstream-charter]{mathdesign}
\usepackage[activate={true, nocompatibility}, final, tracking=true, kerning=true, spacing=true, factor=1100, stretch=10, shrink=10]{microtype}
%Selbe Schriftart für Überschriften												
\addtokomafont{disposition}{\normalfont\bfseries}

%Zeilenabstand							
\usepackage[onehalfspacing]{setspace}

%Literaturverwaltung
\usepackage[backend=biber, style=authoryear, doi=false, isbn=false, maxcitenames=2]{biblatex}		
%Formatierung für Anführungszeichen
\usepackage[autostyle]{csquotes}
%u.a. in et al. umwandeln
\DefineBibliographyStrings{ngerman}{
   andothers = {{et\,al\adddot}},}
%Formatiert die Zitationsklammern
\renewcommand{\labelnamepunct}{\addcolon\addspace}
\renewcommand{\postnotedelim}{\addcolon\addspace}
\DeclareFieldFormat{postnote}{#1}
\DeclareFieldFormat{multipostnote}{#1}
%Leerzeile zwischen den Einträgen im Literaturverzeichnis
\setlength{\bibitemsep}{0.5\baselineskip plus 0.5\baselineskip}
%Schrägstrich bei zwei Autor*innen
\renewcommand{\multinamedelim}[0]{/}
\renewcommand{\finalnamedelim}[0]{/}
%URL bei Artikel- und Büchereinträgen entfernen
\DeclareSourcemap{
  \maps[datatype=bibtex]{
    \map{
      \pertype{article}
      \pertype{book}
       \step[fieldset=url, null]
    }
  }
}


\addbibresource{HA_SNA.bib}

%Seitenränder
%\usepackage[right=4	cm,]{geometry}

%Paket, um Grafiken einzubinden
\usepackage{graphicx}
\usepackage{float}
%Paket, um Grafiken davon abzuhalten, außerhalb bestimmter grenzen zu floaten [section]: verbietet gleiten außerhalb der Section
%\usepackage[section]{placeins}
%zum Formatieren der Bildunterschriften
%\usepackage{caption}
%\captionsetup{format=plain, indention=0cm, justification=justified, singlelinecheck=off, font={small, singlespacing}}
%\captionsetup[table]{name=Tab. }

%Wird für das R-Paket stargazer benötigt
\usepackage{dcolumn}

%Listen
%\usepackage{enumitem}

%pdf-Dateien einbinden
%\usepackage{pdfpages}

%Anhang ohne Seitenzahl im ToC
\makeatletter
\let\partbackup\l@part
\renewcommand*\l@part[2]{\partbackup{#1}{}}
\makeatother

\clubpenalty = 10000
\widowpenalty = 10000
\displaywidowpenalty = 10000

%Verlinktes ToC
\usepackage[colorlinks, pdfpagelabels, pdfstartview=FitH, bookmarksopen=true, bookmarksnumbered=true, linkcolor=black, urlcolor=black, plainpages=false, hypertexnames=false, citecolor=black]{hyperref}



\begin{document}

\titlehead{\flushleft{University of Copenhagen\\
			Department of Political Science\\
			Social Network Analysis (ASTK18106U)\\
			Yevgeniy Golovchenko
			}}
\author{Ole Fechner, Johannes Kopf}
\title{Structure and Strategy of the Internet Research Agency on Twitter}


\date{\normalsize{Copenhagen, \today}}

\publishers{\flushleft{\normalsize{Boddinstr. 15\hfill Moselstr. 50\\
							 12053 Berlin\hfill 50674 Cologne\\
							 ole.fechner@fu-berlin.de\hfill jkopf@posteo.de\\
							 MA Political Science\hfill MA Political Science}}}

\maketitle[0]

\newpage

\thispagestyle{empty}
\tableofcontents

\newpage
\setcounter{page}{1}

\section{Introduction}

\section{Agenda-building and astroturfing}

The process of trying to move an actors agenda to the agenda of other actor's, especially policymakers, is defined as \textit{agenda-building} (\cite[3]{linvill_troll_2018}). This can also be extended to the question of how the public views certain issues, usually by analyzing media coverage of those issues: ``Agenda-building research examines how certain groups, such as those in politics and business, influence what issues journalists cover as well as how the public views issues'' (\cite[434]{parmelee_agenda-building_2014}). Since the rise of social media platforms like Twitter and Facebook, agenda-building takes place in those environments. This is due to journalists drawing heavily on Twitter for their job and, on the other hand, research shows that Twitter is the most popular social media platform for participating in political discussions, which from there are often taken to other media (\cite[435, 437]{parmelee_agenda-building_2014}). Influencing the citizens of another country through the use of media is nothing new, rather it is regularity used in conflicts or during war. ``However, Russia's work on social media has taken agenda-building efforts by nations into a new context'' (\cite[3]{linvill_troll_2018}).

Closely linked to agenda-building is a second phenomenon called (political or/and online) \textit{astroturfing}, which can be characterized as the ``creation of a false or exaggerated impression of grassroots support'' (\cite{harcup_astroturfing_2014}). It describes the strategic and coordinated approach of a group with the aim to create the impression of a certain public opinion, that might not exist in that way. On social media, those groups use many different accounts that post and interact with regular users to create the desired impression. For the purpose of this paper, we see astroturfing as a strategy of agenda-building. The anonymity provided by platforms like Twitter, as well as the covert structure of those groups, make them very hard to discover (\cite[564]{yang_how_2017}). This exploratory study will therefore take the structure of the IRA as a staring point, instead of the presumed agenda behind the organization. Social network analysis, which will be introduced in the next section, offers an excellent tool box to conduct this task.


\section{Social network analysis of the IRA retweet network}

To analyze the structure and possible strategies of the IRA, we will take social network analysis (SNA) as the method of choice. The data-driven character of SNA -- the notion of people relating to each other, and the significance put into the structure and strength of those relations being almost the only one (\cite[982]{golovchenko_state_2018}) -- makes it especially viable for our exploratory approach. SNA conceptualizes the linkages between actors as \enquote{channels for transfer or \enquote{flow} of resources}, while the actors themselves are seen as ``interdependent rather than independent, autonomous units'' (\cite[4]{wasserman_social_1994}). Therefore, the different actor attributes are seen as emerging out of their relations and not vice versa, as in most other quantitative analyses (\cite[8]{wasserman_social_1994}). This allows us to ignore the question of the actual people behind different accounts and tell something about the IRA and its structure, as well as its strategy, as a whole. The most important relation to analyze how the IRA uses astroturfing as a strategy for agenda-building is the distribution of information, in this case via retweets. Accordingly, the foundation of this paper will be a social network consisting of Twitter accounts as nodes and retweets as edges (or linkages).


\subsection{Data: An ego-centered retweet network}

We will use a dataset, published by NBC News (\cite*{popken_twitter_2018}), consisting of 203,451 Tweets by 453 accounts between July 2014 and September 2017, which were linked to the IRA by an official document handed over to US Congress by Twitter. Twitter justifies this linking by referring to “third party sources”, which makes it impossible to reconstruct or evaluate their method. Therefore, we have to assume those accounts’ links to the IRA to be correct, as it is the best evaluation available.

To create the social network, we need to clean the data: First, we drop any of the tweets that are not retweets, which leaves us with 147,428 retweets by 453 troll accounts. Second, 120 trolls, who did not retweet at all, were dropped accordingly. Our dataset now consists a set of 333 unique troll Twitter handles\footnote{A Twitter handle is the screen name of a Twitter account that can be changed by the users.}, who are retweeting others. A variable stating the unique User ID is used to validate uniqueness of the users, showing that there are indeed no duplicate User IDs in the dataset. Since, we only have information about who these 333 accounts did retweet, but not by whom they were retweeted, the network is ego-centered around the group of 333 trolls. Of these 333 trolls, 151 (ca 45\%) retweeted others and were themselves retweeted by other trolls, thus being both sender and receiver. 182 (ca 55\%) trolls only retweeted others, but were not retweeted themselves, making them only senders.\footnote{We define a retweet sender as the person retweeting an original Tweet by another person, who, accordingly, is the retweet receiver.} A third group of 71 trolls was found by looking at who was retweeted by the original group of trolls in the data, thus increasing the number of trolls in our data to a group of 404. These 71 trolls were retweeted by others, though did not send retweets themselves. Finally, the big body of users in the dataset consists of 36,485 users, who are retweeted by the trolls, but are not themselves categorized as trolls by Twitter. Overall, there are 36,889 unique twitter users in the dataset, 404 classified trolls and 36,485 non-trolls. The retweets contain relational information about one user retweeting another, therefore creating a directional edge between the two. Thus, the graph is a directed, ego-centered network of 404 IRA accounts with 36,889 unique Twitter accounts as nodes and edges representing retweets from the sender to the receiver.

To further extend our data and our scope of analysis, we are adding additional qualitative information on the IRA trolls, provided by Darren Linvill and Patrick Lee Warren (\cite*{linvill_troll_2018}) via the online news outlet FiveThirtyEight (citation). Linvill and Warren conduct a qualitative analysis, categorizing a sample of 1,133 IRA troll accounts by examining the Tweets’ content and the account names, applying a temporal analysis of the trolls tweeting behavior after. They “identified five categories of IRA-associated Twitter handles, each with unique patterns of behaviors: \textit{Right Troll}, \textit{Left Troll}, \textit{Newsfeed}, \textit{Hashtag Gamer}, and \textit{Fearmonger}.“ (\cite[6]{linvill_troll_2018}). In addition, there are three categories, which are not used within their analysis, those being \textit{Non-English}, \textit{Commercial} and \textit{Unknown}. The categories \textit{Right Troll} and \textit{Left Troll} need little explanation, as they include users who broadcasted right-leaning populist and socially liberal messages. \textit{Hashtag Gamers} are users who are playing word games on Twitter, mostly non-political, though sometimes including left- or right-leaning messages. \textit{Newsfeed} Trolls are posing as local US News Agencies, mostly linking to legitimate news content, often with a pro-Russian perspective. Fearmongers spread news of crisis events such as Tweets about salmonella infections. The \textit{Non-English} troll category includes users who tweeted in other languages than English, predominately Russian, some German and little French and Spanish. \textit{Commercial} Trolls are not included in our dataset. Finally, users were categorized as \textit{Unknown}, if they could not be assigned to other categories for lack of information in their tweets. These categories will be included in this paper's analysis, since might be interesting to see how they interact with the other trolls. We are appending Linvill and Warren's account categories to our data, finding categories are available for 394 of the 404 troll handles in our dataset, meaning that around 98\% of the trolls in our dataset are categorized.
 
As another attribute, we are appending information on the count of followers of the troll accounts from a second dataset provided by NBC news. The information on the count of followers provided only includes one figure and does not vary over time, without specification of when these follower counts were obtained. We will assume that they are at least to some degree representative and use them as a heuristic. 

Lastly, we are using the full time period of retweets, from July 2014 to September 2017. This paper is not interested in a time period preceding a specific event, like an election or a specific trending discussion, but rather strategic behavior of the trolls in general. That is why it seems to be the right approach to include all of the Tweets in the analysis.


% Table created by stargazer v.5.2 by Marek Hlavac, Harvard University. E-mail: hlavac at fas.harvard.edu
\begin{table}[!htbp] \centering 
  \caption{} 
  \label{} 
\begin{tabular}{@{\extracolsep{5pt}} cccc} 
\\[-1.8ex]\hline 
\hline \\[-1.8ex] 
User & N & Senders & Receivers \\ 
\hline \\[-1.8ex] 
Troll & 404 & 182 & 222 \\ 
Non-Troll & 36,485 & 0 & 36,485 \\
\hline \\[-1.8ex]  
Total & 36,889 & 333 & 36,707 \\ 
\hline \\[-1.8ex] 
\end{tabular} 
\end{table} 

% Table created by stargazer v.5.2 by Marek Hlavac, Harvard University. E-mail: hlavac at fas.harvard.edu
\begin{table}[!htbp] \centering 
  \caption{Troll Statistics} 
  \label{} 
\begin{tabular}{@{\extracolsep{5pt}} D{.}{.}{-3} D{.}{.}{-3} D{.}{.}{-3} D{.}{.}{-3} D{.}{.}{-3} }
\\[-1.8ex]\hline 
\hline \\[-1.8ex] 
\multicolumn{1}{c}{Category} & \multicolumn{1}{c}{N} & \multicolumn{1}{c}{Senders} & \multicolumn{1}{c}{Receivers} & \multicolumn{1}{c}{Average\_Followers} \\ 
\hline \\[-1.8ex] 
\multicolumn{1}{c}{Right} & \multicolumn{1}{c}{101} & \multicolumn{1}{c}{75} & \multicolumn{1}{c}{90} & \multicolumn{1}{c}{4649} \\ 
\multicolumn{1}{c}{Left} & \multicolumn{1}{c}{110} & \multicolumn{1}{c}{104} & \multicolumn{1}{c}{48} & \multicolumn{1}{c}{1783} \\ 
\multicolumn{1}{c}{Hashtag Gamer} & \multicolumn{1}{c}{61} & \multicolumn{1}{c}{43} & \multicolumn{1}{c}{60} & \multicolumn{1}{c}{3021} \\ 
\multicolumn{1}{c}{Non-English} & \multicolumn{1}{c}{106} & \multicolumn{1}{c}{100} & \multicolumn{1}{c}{7} & \multicolumn{1}{c}{2127} \\ 
\multicolumn{1}{c}{Newsfeed} & \multicolumn{1}{c}{11} & \multicolumn{1}{c}{1} & \multicolumn{1}{c}{10} & \multicolumn{1}{c}{16446} \\ 
\multicolumn{1}{c}{Fearmonger} & \multicolumn{1}{c}{4} & \multicolumn{1}{c}{0} & \multicolumn{1}{c}{4} & \multicolumn{1}{c}{0} \\ 
\multicolumn{1}{c}{Unknown} & \multicolumn{1}{c}{11} & \multicolumn{1}{c}{10} & \multicolumn{1}{c}{3} & \multicolumn{1}{c}{3306} \\ 
\multicolumn{1}{c}{Total} & \multicolumn{1}{c}{404} & \multicolumn{1}{c}{333} & \multicolumn{1}{c}{222} & \multicolumn{1}{c}{4476} \\ 
\hline \\[-1.8ex] 
\end{tabular} 
\end{table}

\subsection{Methods}

To understand the IRA's division of labor between the accounts, we begin with testing weather Linvill and Warren's qualitative findings of the different account types would be reproducible via using network properties. For this, we will draw on a community detection algorithm based on modularity, known as the Louvain Method (\cite{blondel_fast_2008}).\footnote{For our network, this algorithm produced the best results. The Infomap algorithm (\cite[cf.][]{rosvall_maps_2008}) results in one big community containing ca 98\% of the nodes.} Communities are mesoscopic structures of a graph, that consist ``of a group of nodes that are relatively densely connected to each other but sparsely connected to other dense groups in the network'' (\cite[1083]{porter_communities_2009}). The Louvain algorithm is based on modularity, ``a scalar value between -1 and 1 that measures the density of links inside communities as compared to links between communities'' (\cites[2]{blondel_fast_2008}[cf. also][1089]{porter_communities_2009}). It approximately maximizes the modularity for each node, thus identifying communities. Finally we will compare the communities with Linvill and Warren's account types, to see weather we can better understand the identified community structure.

One important measure for the cooperation of the accounts is graph density, the proportion of all possible edges that are present in the graph (\cite[101]{wasserman_social_1994}). Since the network is ego-centered, it only makes sense to calculate density for the troll subgraph. We will then compare densities of the different communities, which can give an indication weather the IRA has distinctive strategies for those groups. Following Wassermann and Faust (\cite*[102]{wasserman_social_1994}) we calculate the density of the different community subgraphs $ \Delta_{c} $ as: $$ \Delta_{c} = \frac{2L_{c}}{g_{c}(g_{c} - 1)} ,$$ where $ L_{c} $ is the number of edges present in the community subgraph, and $ g_{s} $ is the number of nodes in said graph. In communities with higher density, the accounts are working more together, meaning they spread information further through retweeting, while accounts in less dense communities rely more on information they put into the network themselves.

Identifying the most important accounts helps to understand the structure of the IRA. Operationalizing importance in SNA is usually done through centrality and prestige indices, whereas not only the chosen actors are considered prestigious, but also those doing the choosing (\cite[170]{wasserman_social_1994}). Since the network is ego-centered around the troll accounts, the indices have limited reach in their interpretation, but still deliver important insights. To show the most central accounts, we will calculate outdegree. The index can only be computed for IRA accounts, since they are the only retweeting in our dataset. We will calculate the outdegree weighted and unweighted respectively, the former results in the accounts that retweet the majority of different accounts, the latter in those retweeting the most.\footnote{A weighted graph is a graph in which each edge carries a value (\cite[140]{wasserman_social_1994}), in this case the number of retweets between two accounts. In an unweighted graph, each edge has the value 1.} The outdegree of a node $ d_{out}(n_{i}) $ is computed as $$ d_{out}(n_{i}) = \sum_	{j} x_{ij} ,$$ where $ x_{ij} $ is an edge from $ i $ to $ j $(\cite[cf.][178]{wasserman_social_1994}. For the weighted outdegree $ x_{ij} $ is multiplied by the value of the edge. Prestige, on the other hand, will be calculated through indegree, which results in those accounts retweeted the most and can therefore tell about what accounts the IRA draws on. It will also be computed for the weighted and unweighted graph separately, which shows the accounts retweeted by the majority of different accounts and those retweeted the most in general respectively. The indegree of a node $ d_{in}(n_{i}) $ is computed simply as $$ d_{in}(n_{i}) = \sum_{j} x_{ji} ,$$ where $ x_{ji} $ is an edge from $ j $ to $ i $ (\cite[cf.][202]{wasserman_social_1994}. For the weighted outdegree $ x_{ji} $ is multiplied by the value of the edge.


\section{Discussion}

\section{Conclusion}

\section{Code (not included in word count)}



\newpage

\printbibliography


\end{document}
