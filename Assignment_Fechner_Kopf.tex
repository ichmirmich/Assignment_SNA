\documentclass[12pt, titlepage=true, toc=bib]{scrartcl}


%Deutsche Silbentrennung
%Trennung von Wörtern mit Umlauten
%Deutsche Sprachregeln
\usepackage[utf8]{inputenc}
\usepackage[T1]{fontenc}
\usepackage[english]{babel}

%Schriftart
\usepackage[bitstream-charter]{mathdesign}
\usepackage[activate={true, nocompatibility}, final, tracking=true, kerning=true, spacing=true, factor=1100, stretch=10, shrink=10]{microtype}
%Selbe Schriftart für Überschriften												
\addtokomafont{disposition}{\normalfont\bfseries}

%Zeilenabstand							
\usepackage[onehalfspacing]{setspace}

%Literaturverwaltung
\usepackage[backend=biber, style=authoryear, doi=false, isbn=false, maxcitenames=2]{biblatex}		
%Formatierung für Anführungszeichen
\usepackage[babel,german=quotes]{csquotes}
%u.a. in et al. umwandeln
\DefineBibliographyStrings{ngerman}{
   andothers = {{et\,al\adddot}},}
%Formatiert die Zitationsklammern
\renewcommand{\labelnamepunct}{\addcolon\addspace}
\renewcommand{\postnotedelim}{\addcolon\addspace}
\DeclareFieldFormat{postnote}{#1}
\DeclareFieldFormat{multipostnote}{#1}
%Leerzeile zwischen den Einträgen im Literaturverzeichnis
\setlength{\bibitemsep}{0.5\baselineskip plus 0.5\baselineskip}
%Schrägstrich bei zwei Autor*innen
\renewcommand{\multinamedelim}[0]{/}
\renewcommand{\finalnamedelim}[0]{/}
%URL bei Artikel- und Büchereinträgen entfernen
\DeclareSourcemap{
  \maps[datatype=bibtex]{
    \map{
      \pertype{article}
      \pertype{book}
       \step[fieldset=url, null]
    }
  }
}


\addbibresource{HA_SNA.bib}

%Seitenränder
%\usepackage[right=4	cm,]{geometry}

%Paket, um Grafiken einzubinden
%\usepackage{graphicx}
%\usepackage{float}
%Paket, um Grafiken davon abzuhalten, außerhalb bestimmter grenzen zu floaten [section]: verbietet gleiten außerhalb der Section
%\usepackage[section]{placeins}
%zum Formatieren der Bildunterschriften
%\usepackage{caption}
%\captionsetup{format=plain, indention=0cm, justification=justified, singlelinecheck=off, font={small, singlespacing}}
%\captionsetup[table]{name=Tab. }

%Listen
%\usepackage{enumitem}

%pdf-Dateien einbinden
%\usepackage{pdfpages}

%Anhang ohne Seitenzahl im ToC
\makeatletter
\let\partbackup\l@part
\renewcommand*\l@part[2]{\partbackup{#1}{}}
\makeatother

\clubpenalty = 10000
\widowpenalty = 10000
\displaywidowpenalty = 10000

%Verlinktes ToC
\usepackage[colorlinks, pdfpagelabels, pdfstartview=FitH, bookmarksopen=true, bookmarksnumbered=true, linkcolor=black, urlcolor=black, plainpages=false, hypertexnames=false, citecolor=black]{hyperref}



\begin{document}

\titlehead{\flushleft{University of Copenhagen\\
			Department of Political Science\\
			Social Network Analysis (ASTK18106U)\\
			Yevgeniy Golovchenko
			}}
\author{Ole Fechner, Johannes Kopf}
\title{Title}


\date{\normalsize{Copenhagen, \today}}

\publishers{\flushleft{\normalsize{Boddinstr. 15\\
									12053 Berlin\\
									ole.fechner@fu-berlin.de\\
									MA Politcal Science}}}

\maketitle[0]

\newpage

\thispagestyle{empty}
\tableofcontents

\newpage
\setcounter{page}{1}

\section{Introduction}

\section{Theory}

\section{Methodology/Operationalisation/Research design}

\subsection{Data}

In early 2018, NBC news published a dataset of 203,451 Tweets by 453 troll accounts between July 2014 and September 2017, which were linked to the Russian IRA by an official document handed over to US Congress by Twitter. Twitter justifies this linking by referring to “third party sources”, which makes it possible to reconstruct or evaluate their method. This paper will assume those accounts’ links to the IRA to be correct, as it is the best evaluation there is. We will use a Social Network Analysis approach to investigate the behaviour of these Russian troll accounts, which is why only relational data will be taken into account within our analysis.
 
Retweets contain relational information about one user retweeting another, therefore creating a directional tie between the two. First, we drop any of the Tweets in the dataset that are not retweets, which leaves us with 147,428 Retweets by 333 troll accounts. There were 120 trolls, who did not retweet and were therefore dropped. Each of these Retweets represents a tie between two users and an edge in our graph, as will be shown later. We define a Retweet sender as the person retweeting an original Tweet by another person, who, accordingly, is the Retweet receiver.
 
Our dataset revolves around a set of 333 unique troll Twitter handles, who are retweeting others. A Twitter handle is the screen name of a Twitter account and can be changed by the users. A variable stating the unique User ID is used to validate unique users, showing that there are no duplicate User IDs in the dataset. There is information about who these 333 accounts did retweet, but not by whom they were retweeted. This makes it an ego-centred network around the group of 333 trolls. Of these 333 trolls, 151 (around 45\%) retweeted others and were themselves retweeted by other trolls, thus being both sender and receiver. 182 (around 55\%) trolls only retweeted others, but were not retweeted themselves, making them only senders. A third group of 71 trolls was found by looking at who was retweeted by the original group of trolls in the data, thus increasing the number of trolls in our data to a group of 404. These 71 trolls were retweeted by others, though did not send retweets themselves. Finally, the big body of users in the dataset consists of 36,485 users, who are retweeted by the trolls, but are not themselves categorized as trolls by Twitter. Overall, there are 36,889 unique twitter users in the dataset, 404 classified trolls and 36,485 non-trolls, which together represent 36,889 nodes in our network.

To further extend our data and our scope of analysis, we are adding additional qualitative information on the IRA trolls, provided by Darren Linvill and Patrick Lee Warren (\cite*{linvill_troll_nodate}) via the online news outlet FiveThirtyEight (citation). Linvill and Warren conduct a qualitative analysis, categorizing a sample of 1,133 IRA troll accounts by examining the Tweets’ content and the account names, applying a temporal analysis of the trolls tweeting behaviour after. They “[…] identified five categories of IRA-associated Twitter handles, each with unique patterns of behaviors: \textit{Right Troll}, \textit{Left Troll}, \textit{Newsfeed}, \textit{Hashtag Gamer}, and \textit{Fearmonger}.“ (\cite[6]{linvill_troll_nodate}). In addition, there are three categories, which are not used within their analysis, those being \textit{Non-English}, \textit{Commercial} and \textit{Unknown}. The categories \textit{Right Troll} and \textit{Left Troll} need little explanation, as they include users who broadcasted right-leaning populist and socially libral messages. \textit{Hashtag Gamers} are users who are playing word games on Twitter, mostly non-political, though sometimes including left- or right-leaning messages. \textit{Newsfeed} Trolls are posing as local US News Agencies, mostly linking to legitimate news content, often with a pro-Russian perspective. Fearmongers spread news of crisis events such as Tweets about salmonella infections. The \textit{Non-English} troll category includes users who tweeted in other languages than English, predominately Russian, some German and little French and Spanish. \textit{Commercial} Trolls are not included in our dataset. Finally, users were categorized as \textit{Unknown}, if they could not be assigned to other categories for lack of information in their tweets. These categories will be included in this paper’s analysis, since might be interesting to see how they interact with the other trolls. We are appending Linvill and Warren’s account categories to our data, finding categories are available for 394 of the 404 troll handles in our dataset, meaning that around 98\% of the trolls in our dataset are categorized.
 
As another attribute, we are appending information on the count of followers of the troll accounts from a second dataset provided by NBC news. The information on the count of followers provided only includes one figure and does not vary over time, without specification of when these follower counts were obtained. We will assume that they are at least to some degree representative and use them as a heuristic. 

Lastly, we are using the full time period of Retweets, from July 2014 to September 2017. This paper is not interested in a time period preceding a specific event, like an election or a specific trending discussion, but rather strategic behaviour of the trolls in general. That is why it seems to be the right approach to include all of the Tweets in the analysis.

\begin{table}[ht]
\centering
\begin{tabular}{llll}
  \hline
User & N & Senders & Receivers \\ 
  \hline
Troll & 404 & 182 & 222 \\ 
  Non-Troll & 36,485 & 0 & 36,485 \\ 
  Total & 36,889 & 333 & 36,707 \\ 
   \hline
\end{tabular}
\end{table}

% Table created by stargazer v.5.2 by Marek Hlavac, Harvard University. E-mail: hlavac at fas.harvard.edu
\begin{table}[!htbp] \centering 
  \caption{} 
  \label{} 
\begin{tabular}{@{\extracolsep{5pt}} cccc} 
\\[-1.8ex]\hline 
\hline \\[-1.8ex] 
User & N & Senders & Receivers \\ 
\hline \\[-1.8ex] 
Troll & 404 & 182 & 222 \\ 
Non-Troll & 36,485 & 0 & 36,485 \\
\hline \\[-1.8ex]  
Total & 36,889 & 333 & 36,707 \\ 
\hline \\[-1.8ex] 
\end{tabular} 
\end{table} 

% Table created by stargazer v.5.2 by Marek Hlavac, Harvard University. E-mail: hlavac at fas.harvard.edu
\begin{table}[!htbp] \centering 
  \caption{Troll Statistics} 
  \label{} 
\begin{tabular}{@{\extracolsep{5pt}} D{.}{.}{-3} D{.}{.}{-3} D{.}{.}{-3} D{.}{.}{-3} D{.}{.}{-3} } 
\\[-1.8ex]\hline 
\hline \\[-1.8ex] 
\multicolumn{1}{c}{Category} & \multicolumn{1}{c}{N} & \multicolumn{1}{c}{Senders} & \multicolumn{1}{c}{Receivers} & \multicolumn{1}{c}{Average\_Followers} \\ 
\hline \\[-1.8ex] 
\multicolumn{1}{c}{Right} & \multicolumn{1}{c}{101} & \multicolumn{1}{c}{75} & \multicolumn{1}{c}{90} & \multicolumn{1}{c}{4649} \\ 
\multicolumn{1}{c}{Left} & \multicolumn{1}{c}{110} & \multicolumn{1}{c}{104} & \multicolumn{1}{c}{48} & \multicolumn{1}{c}{1783} \\ 
\multicolumn{1}{c}{Hashtag Gamer} & \multicolumn{1}{c}{61} & \multicolumn{1}{c}{43} & \multicolumn{1}{c}{60} & \multicolumn{1}{c}{3021} \\ 
\multicolumn{1}{c}{Non-English} & \multicolumn{1}{c}{106} & \multicolumn{1}{c}{100} & \multicolumn{1}{c}{7} & \multicolumn{1}{c}{2127} \\ 
\multicolumn{1}{c}{Newsfeed} & \multicolumn{1}{c}{11} & \multicolumn{1}{c}{1} & \multicolumn{1}{c}{10} & \multicolumn{1}{c}{16446} \\ 
\multicolumn{1}{c}{Fearmonger} & \multicolumn{1}{c}{4} & \multicolumn{1}{c}{0} & \multicolumn{1}{c}{4} & \multicolumn{1}{c}{0} \\ 
\multicolumn{1}{c}{Unknown} & \multicolumn{1}{c}{11} & \multicolumn{1}{c}{10} & \multicolumn{1}{c}{3} & \multicolumn{1}{c}{3306} \\ 
\multicolumn{1}{c}{Total} & \multicolumn{1}{c}{404} & \multicolumn{1}{c}{333} & \multicolumn{1}{c}{222} & \multicolumn{1}{c}{4476} \\ 
\hline \\[-1.8ex] 
\end{tabular} 
\end{table}

\section{Methods}

\section{Discussion}

\section{Conclusion}

\section{Code (not included in word count)}



\newpage

\printbibliography


\end{document}
